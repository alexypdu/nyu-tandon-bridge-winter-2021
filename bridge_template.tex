\documentclass[11pt]{article}
\usepackage{fullpage}
\usepackage{amsmath,amsfonts,amsthm,amssymb}
\usepackage{url}
\usepackage[demo]{graphicx}
\usepackage{caption} 
\usepackage{algpseudocode}
\usepackage{bbm}
\usepackage{float}
\usepackage{framed}
\usepackage{enumerate}
\usepackage{color}
\usepackage[colorlinks=true, linkcolor=red, urlcolor=blue, citecolor=blue]{hyperref}

\DeclareMathOperator*{\E}{\mathbb{E}}
\let\Pr\relax
\DeclareMathOperator*{\Pr}{\mathbb{P}}
\DeclareMathOperator*{\R}{\mathbb{E}}

\topmargin 0pt
\advance \topmargin by -\headheight
\advance \topmargin by -\headsep
\textheight 8.9in
\oddsidemargin 0pt
\evensidemargin \oddsidemargin
\marginparwidth 0.5in
\textwidth 6.5in

\parindent 0in
\parskip 1.5ex

\newcommand{\homework}[2]{
	\noindent
	\begin{center}
		\framebox{
			\vbox{
				\hbox to 6.50in { {\bf NYU Computer Science Bridge to Tandon Course} \hfill Winter 2021 }
				\vspace{4mm}
				\hbox to 6.50in { {\Large \hfill Homework #1  \hfill} }
				\vspace{2mm}
				\hbox to 6.50in { {Name: #2 \hfill} }
			}
		}
	\end{center}
	\vspace*{4mm}
}

\begin{document}
	
	\homework{1}{Star Student}
	
	\section*{Problem 1}
	\textbf{Problem statement:} Problem statement here
	\medskip
	
	Problem answer
	
	\begin{align*}
	\E[X + Y]  &= \E[X] + \E[Y] \\
	& = 42.
	\end{align*}
	
	\begin{figure}[h]
		\centering
		\includegraphics[width=0.3\textwidth]{image.jpg}
		\caption{Images can be included by saving in the same directory as your source code or uploading to an online portal.  These can include hand-drawn images or snippets.  \textbf{IMAGES SHOULD NOT BE SCREENSHOTS OF HANDWRITTEN ANSWERS THAT COULD OTHERWISE BE WRITTEN IN LaTeX}!}
	\end{figure}
	
	You might find it useful to include code snippets:
	\begin{verbatim}
	class majorityElement(self, nums) {
	    count = 0;
	    candidate = NULL;
    	
    	for (i = 1; i < nums; i++) {
        	if (count == 0)
        	    candidate = i;
        	if (i == candidate)
        	    count++;
        	else
        	    count =- 1;
    	}
    	
	    return candidate;
	}
	\end{verbatim}
	
	And sometimes it might be helpful to include pseudocode:
	
	\begin{algorithmic}
		\State $count \gets 0$
		\State $candidate \gets null$
		
		\For {$i \in {a_1,\dots,a_n}$}
		\If {$count == 0$}
		\State $candidate \gets a_i$
		\EndIf
		\State {$count \gets count + 1$}
		\EndFor	
		\State \Return $candidate$
	\end{algorithmic}
	
	
	\newpage
	\section*{Problem 2}
	\textbf{Problem statement:} Problem statement here
	\medskip
	
	Please start each new problem on a new page. This allows you to tag questions individually for grading.  Sub-problems (e.g. 1(a), 1(b), etc.) can be included on the same page or without breaking to a new page.  
	
\end{document}